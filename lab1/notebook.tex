
% Default to the notebook output style

    


% Inherit from the specified cell style.




    
\documentclass[11pt]{article}

    
    
    \usepackage[T1]{fontenc}
    % Nicer default font (+ math font) than Computer Modern for most use cases
    \usepackage{mathpazo}

    % Basic figure setup, for now with no caption control since it's done
    % automatically by Pandoc (which extracts ![](path) syntax from Markdown).
    \usepackage{graphicx}
    % We will generate all images so they have a width \maxwidth. This means
    % that they will get their normal width if they fit onto the page, but
    % are scaled down if they would overflow the margins.
    \makeatletter
    \def\maxwidth{\ifdim\Gin@nat@width>\linewidth\linewidth
    \else\Gin@nat@width\fi}
    \makeatother
    \let\Oldincludegraphics\includegraphics
    % Set max figure width to be 80% of text width, for now hardcoded.
    \renewcommand{\includegraphics}[1]{\Oldincludegraphics[width=.8\maxwidth]{#1}}
    % Ensure that by default, figures have no caption (until we provide a
    % proper Figure object with a Caption API and a way to capture that
    % in the conversion process - todo).
    \usepackage{caption}
    \DeclareCaptionLabelFormat{nolabel}{}
    \captionsetup{labelformat=nolabel}

    \usepackage{adjustbox} % Used to constrain images to a maximum size 
    \usepackage{xcolor} % Allow colors to be defined
    \usepackage{enumerate} % Needed for markdown enumerations to work
    \usepackage{geometry} % Used to adjust the document margins
    \usepackage{amsmath} % Equations
    \usepackage{amssymb} % Equations
    \usepackage{textcomp} % defines textquotesingle
    % Hack from http://tex.stackexchange.com/a/47451/13684:
    \AtBeginDocument{%
        \def\PYZsq{\textquotesingle}% Upright quotes in Pygmentized code
    }
    \usepackage{upquote} % Upright quotes for verbatim code
    \usepackage{eurosym} % defines \euro
    \usepackage[mathletters]{ucs} % Extended unicode (utf-8) support
    \usepackage[utf8x]{inputenc} % Allow utf-8 characters in the tex document
    \usepackage{fancyvrb} % verbatim replacement that allows latex
    \usepackage{grffile} % extends the file name processing of package graphics 
                         % to support a larger range 
    % The hyperref package gives us a pdf with properly built
    % internal navigation ('pdf bookmarks' for the table of contents,
    % internal cross-reference links, web links for URLs, etc.)
    \usepackage{hyperref}
    \usepackage{longtable} % longtable support required by pandoc >1.10
    \usepackage{booktabs}  % table support for pandoc > 1.12.2
    \usepackage[inline]{enumitem} % IRkernel/repr support (it uses the enumerate* environment)
    \usepackage[normalem]{ulem} % ulem is needed to support strikethroughs (\sout)
                                % normalem makes italics be italics, not underlines
    

    
    
    % Colors for the hyperref package
    \definecolor{urlcolor}{rgb}{0,.145,.698}
    \definecolor{linkcolor}{rgb}{.71,0.21,0.01}
    \definecolor{citecolor}{rgb}{.12,.54,.11}

    % ANSI colors
    \definecolor{ansi-black}{HTML}{3E424D}
    \definecolor{ansi-black-intense}{HTML}{282C36}
    \definecolor{ansi-red}{HTML}{E75C58}
    \definecolor{ansi-red-intense}{HTML}{B22B31}
    \definecolor{ansi-green}{HTML}{00A250}
    \definecolor{ansi-green-intense}{HTML}{007427}
    \definecolor{ansi-yellow}{HTML}{DDB62B}
    \definecolor{ansi-yellow-intense}{HTML}{B27D12}
    \definecolor{ansi-blue}{HTML}{208FFB}
    \definecolor{ansi-blue-intense}{HTML}{0065CA}
    \definecolor{ansi-magenta}{HTML}{D160C4}
    \definecolor{ansi-magenta-intense}{HTML}{A03196}
    \definecolor{ansi-cyan}{HTML}{60C6C8}
    \definecolor{ansi-cyan-intense}{HTML}{258F8F}
    \definecolor{ansi-white}{HTML}{C5C1B4}
    \definecolor{ansi-white-intense}{HTML}{A1A6B2}

    % commands and environments needed by pandoc snippets
    % extracted from the output of `pandoc -s`
    \providecommand{\tightlist}{%
      \setlength{\itemsep}{0pt}\setlength{\parskip}{0pt}}
    \DefineVerbatimEnvironment{Highlighting}{Verbatim}{commandchars=\\\{\}}
    % Add ',fontsize=\small' for more characters per line
    \newenvironment{Shaded}{}{}
    \newcommand{\KeywordTok}[1]{\textcolor[rgb]{0.00,0.44,0.13}{\textbf{{#1}}}}
    \newcommand{\DataTypeTok}[1]{\textcolor[rgb]{0.56,0.13,0.00}{{#1}}}
    \newcommand{\DecValTok}[1]{\textcolor[rgb]{0.25,0.63,0.44}{{#1}}}
    \newcommand{\BaseNTok}[1]{\textcolor[rgb]{0.25,0.63,0.44}{{#1}}}
    \newcommand{\FloatTok}[1]{\textcolor[rgb]{0.25,0.63,0.44}{{#1}}}
    \newcommand{\CharTok}[1]{\textcolor[rgb]{0.25,0.44,0.63}{{#1}}}
    \newcommand{\StringTok}[1]{\textcolor[rgb]{0.25,0.44,0.63}{{#1}}}
    \newcommand{\CommentTok}[1]{\textcolor[rgb]{0.38,0.63,0.69}{\textit{{#1}}}}
    \newcommand{\OtherTok}[1]{\textcolor[rgb]{0.00,0.44,0.13}{{#1}}}
    \newcommand{\AlertTok}[1]{\textcolor[rgb]{1.00,0.00,0.00}{\textbf{{#1}}}}
    \newcommand{\FunctionTok}[1]{\textcolor[rgb]{0.02,0.16,0.49}{{#1}}}
    \newcommand{\RegionMarkerTok}[1]{{#1}}
    \newcommand{\ErrorTok}[1]{\textcolor[rgb]{1.00,0.00,0.00}{\textbf{{#1}}}}
    \newcommand{\NormalTok}[1]{{#1}}
    
    % Additional commands for more recent versions of Pandoc
    \newcommand{\ConstantTok}[1]{\textcolor[rgb]{0.53,0.00,0.00}{{#1}}}
    \newcommand{\SpecialCharTok}[1]{\textcolor[rgb]{0.25,0.44,0.63}{{#1}}}
    \newcommand{\VerbatimStringTok}[1]{\textcolor[rgb]{0.25,0.44,0.63}{{#1}}}
    \newcommand{\SpecialStringTok}[1]{\textcolor[rgb]{0.73,0.40,0.53}{{#1}}}
    \newcommand{\ImportTok}[1]{{#1}}
    \newcommand{\DocumentationTok}[1]{\textcolor[rgb]{0.73,0.13,0.13}{\textit{{#1}}}}
    \newcommand{\AnnotationTok}[1]{\textcolor[rgb]{0.38,0.63,0.69}{\textbf{\textit{{#1}}}}}
    \newcommand{\CommentVarTok}[1]{\textcolor[rgb]{0.38,0.63,0.69}{\textbf{\textit{{#1}}}}}
    \newcommand{\VariableTok}[1]{\textcolor[rgb]{0.10,0.09,0.49}{{#1}}}
    \newcommand{\ControlFlowTok}[1]{\textcolor[rgb]{0.00,0.44,0.13}{\textbf{{#1}}}}
    \newcommand{\OperatorTok}[1]{\textcolor[rgb]{0.40,0.40,0.40}{{#1}}}
    \newcommand{\BuiltInTok}[1]{{#1}}
    \newcommand{\ExtensionTok}[1]{{#1}}
    \newcommand{\PreprocessorTok}[1]{\textcolor[rgb]{0.74,0.48,0.00}{{#1}}}
    \newcommand{\AttributeTok}[1]{\textcolor[rgb]{0.49,0.56,0.16}{{#1}}}
    \newcommand{\InformationTok}[1]{\textcolor[rgb]{0.38,0.63,0.69}{\textbf{\textit{{#1}}}}}
    \newcommand{\WarningTok}[1]{\textcolor[rgb]{0.38,0.63,0.69}{\textbf{\textit{{#1}}}}}
    
    
    % Define a nice break command that doesn't care if a line doesn't already
    % exist.
    \def\br{\hspace*{\fill} \\* }
    % Math Jax compatability definitions
    \def\gt{>}
    \def\lt{<}
    % Document parameters
    \title{Lab 1 - Problem 1}
    
    
    

    % Pygments definitions
    
\makeatletter
\def\PY@reset{\let\PY@it=\relax \let\PY@bf=\relax%
    \let\PY@ul=\relax \let\PY@tc=\relax%
    \let\PY@bc=\relax \let\PY@ff=\relax}
\def\PY@tok#1{\csname PY@tok@#1\endcsname}
\def\PY@toks#1+{\ifx\relax#1\empty\else%
    \PY@tok{#1}\expandafter\PY@toks\fi}
\def\PY@do#1{\PY@bc{\PY@tc{\PY@ul{%
    \PY@it{\PY@bf{\PY@ff{#1}}}}}}}
\def\PY#1#2{\PY@reset\PY@toks#1+\relax+\PY@do{#2}}

\expandafter\def\csname PY@tok@vm\endcsname{\def\PY@tc##1{\textcolor[rgb]{0.10,0.09,0.49}{##1}}}
\expandafter\def\csname PY@tok@dl\endcsname{\def\PY@tc##1{\textcolor[rgb]{0.73,0.13,0.13}{##1}}}
\expandafter\def\csname PY@tok@na\endcsname{\def\PY@tc##1{\textcolor[rgb]{0.49,0.56,0.16}{##1}}}
\expandafter\def\csname PY@tok@cs\endcsname{\let\PY@it=\textit\def\PY@tc##1{\textcolor[rgb]{0.25,0.50,0.50}{##1}}}
\expandafter\def\csname PY@tok@c1\endcsname{\let\PY@it=\textit\def\PY@tc##1{\textcolor[rgb]{0.25,0.50,0.50}{##1}}}
\expandafter\def\csname PY@tok@go\endcsname{\def\PY@tc##1{\textcolor[rgb]{0.53,0.53,0.53}{##1}}}
\expandafter\def\csname PY@tok@sr\endcsname{\def\PY@tc##1{\textcolor[rgb]{0.73,0.40,0.53}{##1}}}
\expandafter\def\csname PY@tok@gs\endcsname{\let\PY@bf=\textbf}
\expandafter\def\csname PY@tok@no\endcsname{\def\PY@tc##1{\textcolor[rgb]{0.53,0.00,0.00}{##1}}}
\expandafter\def\csname PY@tok@gd\endcsname{\def\PY@tc##1{\textcolor[rgb]{0.63,0.00,0.00}{##1}}}
\expandafter\def\csname PY@tok@vc\endcsname{\def\PY@tc##1{\textcolor[rgb]{0.10,0.09,0.49}{##1}}}
\expandafter\def\csname PY@tok@o\endcsname{\def\PY@tc##1{\textcolor[rgb]{0.40,0.40,0.40}{##1}}}
\expandafter\def\csname PY@tok@nc\endcsname{\let\PY@bf=\textbf\def\PY@tc##1{\textcolor[rgb]{0.00,0.00,1.00}{##1}}}
\expandafter\def\csname PY@tok@s2\endcsname{\def\PY@tc##1{\textcolor[rgb]{0.73,0.13,0.13}{##1}}}
\expandafter\def\csname PY@tok@mb\endcsname{\def\PY@tc##1{\textcolor[rgb]{0.40,0.40,0.40}{##1}}}
\expandafter\def\csname PY@tok@kn\endcsname{\let\PY@bf=\textbf\def\PY@tc##1{\textcolor[rgb]{0.00,0.50,0.00}{##1}}}
\expandafter\def\csname PY@tok@ni\endcsname{\let\PY@bf=\textbf\def\PY@tc##1{\textcolor[rgb]{0.60,0.60,0.60}{##1}}}
\expandafter\def\csname PY@tok@il\endcsname{\def\PY@tc##1{\textcolor[rgb]{0.40,0.40,0.40}{##1}}}
\expandafter\def\csname PY@tok@bp\endcsname{\def\PY@tc##1{\textcolor[rgb]{0.00,0.50,0.00}{##1}}}
\expandafter\def\csname PY@tok@vg\endcsname{\def\PY@tc##1{\textcolor[rgb]{0.10,0.09,0.49}{##1}}}
\expandafter\def\csname PY@tok@sd\endcsname{\let\PY@it=\textit\def\PY@tc##1{\textcolor[rgb]{0.73,0.13,0.13}{##1}}}
\expandafter\def\csname PY@tok@mo\endcsname{\def\PY@tc##1{\textcolor[rgb]{0.40,0.40,0.40}{##1}}}
\expandafter\def\csname PY@tok@kp\endcsname{\def\PY@tc##1{\textcolor[rgb]{0.00,0.50,0.00}{##1}}}
\expandafter\def\csname PY@tok@sb\endcsname{\def\PY@tc##1{\textcolor[rgb]{0.73,0.13,0.13}{##1}}}
\expandafter\def\csname PY@tok@vi\endcsname{\def\PY@tc##1{\textcolor[rgb]{0.10,0.09,0.49}{##1}}}
\expandafter\def\csname PY@tok@ge\endcsname{\let\PY@it=\textit}
\expandafter\def\csname PY@tok@sh\endcsname{\def\PY@tc##1{\textcolor[rgb]{0.73,0.13,0.13}{##1}}}
\expandafter\def\csname PY@tok@gu\endcsname{\let\PY@bf=\textbf\def\PY@tc##1{\textcolor[rgb]{0.50,0.00,0.50}{##1}}}
\expandafter\def\csname PY@tok@gr\endcsname{\def\PY@tc##1{\textcolor[rgb]{1.00,0.00,0.00}{##1}}}
\expandafter\def\csname PY@tok@si\endcsname{\let\PY@bf=\textbf\def\PY@tc##1{\textcolor[rgb]{0.73,0.40,0.53}{##1}}}
\expandafter\def\csname PY@tok@nt\endcsname{\let\PY@bf=\textbf\def\PY@tc##1{\textcolor[rgb]{0.00,0.50,0.00}{##1}}}
\expandafter\def\csname PY@tok@gp\endcsname{\let\PY@bf=\textbf\def\PY@tc##1{\textcolor[rgb]{0.00,0.00,0.50}{##1}}}
\expandafter\def\csname PY@tok@mi\endcsname{\def\PY@tc##1{\textcolor[rgb]{0.40,0.40,0.40}{##1}}}
\expandafter\def\csname PY@tok@kt\endcsname{\def\PY@tc##1{\textcolor[rgb]{0.69,0.00,0.25}{##1}}}
\expandafter\def\csname PY@tok@w\endcsname{\def\PY@tc##1{\textcolor[rgb]{0.73,0.73,0.73}{##1}}}
\expandafter\def\csname PY@tok@mf\endcsname{\def\PY@tc##1{\textcolor[rgb]{0.40,0.40,0.40}{##1}}}
\expandafter\def\csname PY@tok@m\endcsname{\def\PY@tc##1{\textcolor[rgb]{0.40,0.40,0.40}{##1}}}
\expandafter\def\csname PY@tok@ow\endcsname{\let\PY@bf=\textbf\def\PY@tc##1{\textcolor[rgb]{0.67,0.13,1.00}{##1}}}
\expandafter\def\csname PY@tok@fm\endcsname{\def\PY@tc##1{\textcolor[rgb]{0.00,0.00,1.00}{##1}}}
\expandafter\def\csname PY@tok@kd\endcsname{\let\PY@bf=\textbf\def\PY@tc##1{\textcolor[rgb]{0.00,0.50,0.00}{##1}}}
\expandafter\def\csname PY@tok@sx\endcsname{\def\PY@tc##1{\textcolor[rgb]{0.00,0.50,0.00}{##1}}}
\expandafter\def\csname PY@tok@cp\endcsname{\def\PY@tc##1{\textcolor[rgb]{0.74,0.48,0.00}{##1}}}
\expandafter\def\csname PY@tok@cm\endcsname{\let\PY@it=\textit\def\PY@tc##1{\textcolor[rgb]{0.25,0.50,0.50}{##1}}}
\expandafter\def\csname PY@tok@nd\endcsname{\def\PY@tc##1{\textcolor[rgb]{0.67,0.13,1.00}{##1}}}
\expandafter\def\csname PY@tok@gh\endcsname{\let\PY@bf=\textbf\def\PY@tc##1{\textcolor[rgb]{0.00,0.00,0.50}{##1}}}
\expandafter\def\csname PY@tok@sa\endcsname{\def\PY@tc##1{\textcolor[rgb]{0.73,0.13,0.13}{##1}}}
\expandafter\def\csname PY@tok@kr\endcsname{\let\PY@bf=\textbf\def\PY@tc##1{\textcolor[rgb]{0.00,0.50,0.00}{##1}}}
\expandafter\def\csname PY@tok@nn\endcsname{\let\PY@bf=\textbf\def\PY@tc##1{\textcolor[rgb]{0.00,0.00,1.00}{##1}}}
\expandafter\def\csname PY@tok@nl\endcsname{\def\PY@tc##1{\textcolor[rgb]{0.63,0.63,0.00}{##1}}}
\expandafter\def\csname PY@tok@mh\endcsname{\def\PY@tc##1{\textcolor[rgb]{0.40,0.40,0.40}{##1}}}
\expandafter\def\csname PY@tok@gi\endcsname{\def\PY@tc##1{\textcolor[rgb]{0.00,0.63,0.00}{##1}}}
\expandafter\def\csname PY@tok@nv\endcsname{\def\PY@tc##1{\textcolor[rgb]{0.10,0.09,0.49}{##1}}}
\expandafter\def\csname PY@tok@s1\endcsname{\def\PY@tc##1{\textcolor[rgb]{0.73,0.13,0.13}{##1}}}
\expandafter\def\csname PY@tok@ss\endcsname{\def\PY@tc##1{\textcolor[rgb]{0.10,0.09,0.49}{##1}}}
\expandafter\def\csname PY@tok@gt\endcsname{\def\PY@tc##1{\textcolor[rgb]{0.00,0.27,0.87}{##1}}}
\expandafter\def\csname PY@tok@ne\endcsname{\let\PY@bf=\textbf\def\PY@tc##1{\textcolor[rgb]{0.82,0.25,0.23}{##1}}}
\expandafter\def\csname PY@tok@err\endcsname{\def\PY@bc##1{\setlength{\fboxsep}{0pt}\fcolorbox[rgb]{1.00,0.00,0.00}{1,1,1}{\strut ##1}}}
\expandafter\def\csname PY@tok@c\endcsname{\let\PY@it=\textit\def\PY@tc##1{\textcolor[rgb]{0.25,0.50,0.50}{##1}}}
\expandafter\def\csname PY@tok@s\endcsname{\def\PY@tc##1{\textcolor[rgb]{0.73,0.13,0.13}{##1}}}
\expandafter\def\csname PY@tok@k\endcsname{\let\PY@bf=\textbf\def\PY@tc##1{\textcolor[rgb]{0.00,0.50,0.00}{##1}}}
\expandafter\def\csname PY@tok@se\endcsname{\let\PY@bf=\textbf\def\PY@tc##1{\textcolor[rgb]{0.73,0.40,0.13}{##1}}}
\expandafter\def\csname PY@tok@cpf\endcsname{\let\PY@it=\textit\def\PY@tc##1{\textcolor[rgb]{0.25,0.50,0.50}{##1}}}
\expandafter\def\csname PY@tok@kc\endcsname{\let\PY@bf=\textbf\def\PY@tc##1{\textcolor[rgb]{0.00,0.50,0.00}{##1}}}
\expandafter\def\csname PY@tok@nb\endcsname{\def\PY@tc##1{\textcolor[rgb]{0.00,0.50,0.00}{##1}}}
\expandafter\def\csname PY@tok@nf\endcsname{\def\PY@tc##1{\textcolor[rgb]{0.00,0.00,1.00}{##1}}}
\expandafter\def\csname PY@tok@sc\endcsname{\def\PY@tc##1{\textcolor[rgb]{0.73,0.13,0.13}{##1}}}
\expandafter\def\csname PY@tok@ch\endcsname{\let\PY@it=\textit\def\PY@tc##1{\textcolor[rgb]{0.25,0.50,0.50}{##1}}}

\def\PYZbs{\char`\\}
\def\PYZus{\char`\_}
\def\PYZob{\char`\{}
\def\PYZcb{\char`\}}
\def\PYZca{\char`\^}
\def\PYZam{\char`\&}
\def\PYZlt{\char`\<}
\def\PYZgt{\char`\>}
\def\PYZsh{\char`\#}
\def\PYZpc{\char`\%}
\def\PYZdl{\char`\$}
\def\PYZhy{\char`\-}
\def\PYZsq{\char`\'}
\def\PYZdq{\char`\"}
\def\PYZti{\char`\~}
% for compatibility with earlier versions
\def\PYZat{@}
\def\PYZlb{[}
\def\PYZrb{]}
\makeatother


    % Exact colors from NB
    \definecolor{incolor}{rgb}{0.0, 0.0, 0.5}
    \definecolor{outcolor}{rgb}{0.545, 0.0, 0.0}



    
    % Prevent overflowing lines due to hard-to-break entities
    \sloppy 
    % Setup hyperref package
    \hypersetup{
      breaklinks=true,  % so long urls are correctly broken across lines
      colorlinks=true,
      urlcolor=urlcolor,
      linkcolor=linkcolor,
      citecolor=citecolor,
      }
    % Slightly bigger margins than the latex defaults
    
    \geometry{verbose,tmargin=1in,bmargin=1in,lmargin=1in,rmargin=1in}
    
    

    \begin{document}
    
    
    \maketitle
    
    

    
    \section{Lab 1: Markov Decision Processes - Problem
1}\label{lab-1-markov-decision-processes---problem-1}

\subsection{Lab Instructions}\label{lab-instructions}

All your answers should be written in this notebook. You shouldn't need
to write or modify any other files.

\textbf{You should execute every block of code to not miss any
dependency.}

\emph{This project was developed by Peter Chen, Rocky Duan, Pieter
Abbeel for the Berkeley Deep RL Bootcamp, August 2017. Bootcamp website
with slides and lecture videos:
https://sites.google.com/view/deep-rl-bootcamp/. It is adapted from
Berkeley Deep RL Class
\href{https://github.com/berkeleydeeprlcourse/homework/blob/c1027d83cd542e67ebed982d44666e0d22a00141/hw2/HW2.ipynb}{HW2}
\href{https://github.com/berkeleydeeprlcourse/homework/blob/master/LICENSE}{(license)}}

\begin{center}\rule{0.5\linewidth}{\linethickness}\end{center}

    \subsection{Introduction}\label{introduction}

This assignment will review the two classic methods for solving Markov
Decision Processes (MDPs) with finite state and action spaces. We will
implement value iteration (VI) and policy iteration (PI) for a finite
MDP, both of which find the optimal policy in a finite number of
iterations.

The experiments here will use the Frozen Lake environment, a simple
gridworld MDP that is taken from \texttt{gym} and slightly modified for
this assignment. In this MDP, the agent must navigate from the start
state to the goal state on a 4x4 grid, with stochastic transitions.

    \begin{Verbatim}[commandchars=\\\{\}]
{\color{incolor}In [{\color{incolor}2}]:} \PY{k+kn}{from} \PY{n+nn}{misc} \PY{k}{import} \PY{n}{FrozenLakeEnv}\PY{p}{,} \PY{n}{make\PYZus{}grader}
        \PY{n}{env} \PY{o}{=} \PY{n}{FrozenLakeEnv}\PY{p}{(}\PY{p}{)}
        \PY{n+nb}{print}\PY{p}{(}\PY{n}{env}\PY{o}{.}\PY{n+nv+vm}{\PYZus{}\PYZus{}doc\PYZus{}\PYZus{}}\PY{p}{)}
\end{Verbatim}


    \begin{Verbatim}[commandchars=\\\{\}]

    Winter is here. You and your friends were tossing around a frisbee at the park
    when you made a wild throw that left the frisbee out in the middle of the lake.
    The water is mostly frozen, but there are a few holes where the ice has melted.
    If you step into one of those holes, you'll fall into the freezing water.
    At this time, there's an international frisbee shortage, so it's absolutely imperative that
    you navigate across the lake and retrieve the disc.
    However, the ice is slippery, so you won't always move in the direction you intend.
    The surface is described using a grid like the following

        SFFF
        FHFH
        FFFH
        HFFG

    S : starting point, safe
    F : frozen surface, safe
    H : hole, fall to your doom
    G : goal, where the frisbee is located

    The episode ends when you reach the goal or fall in a hole.
    You receive a reward of 1 if you reach the goal, and zero otherwise.

    

    \end{Verbatim}

    Let's look at what a random episode looks like.

    \begin{Verbatim}[commandchars=\\\{\}]
{\color{incolor}In [{\color{incolor}3}]:} \PY{c+c1}{\PYZsh{} Some basic imports and setup}
        \PY{k+kn}{import} \PY{n+nn}{numpy} \PY{k}{as} \PY{n+nn}{np}\PY{o}{,} \PY{n+nn}{numpy}\PY{n+nn}{.}\PY{n+nn}{random} \PY{k}{as} \PY{n+nn}{nr}\PY{o}{,} \PY{n+nn}{gym}
        \PY{k+kn}{import} \PY{n+nn}{matplotlib}\PY{n+nn}{.}\PY{n+nn}{pyplot} \PY{k}{as} \PY{n+nn}{plt}
        \PY{o}{\PYZpc{}}\PY{k}{matplotlib} inline
        \PY{n}{np}\PY{o}{.}\PY{n}{set\PYZus{}printoptions}\PY{p}{(}\PY{n}{precision}\PY{o}{=}\PY{l+m+mi}{3}\PY{p}{)}
        
        \PY{c+c1}{\PYZsh{} Seed RNGs so you get the same printouts as me}
        \PY{n}{env}\PY{o}{.}\PY{n}{seed}\PY{p}{(}\PY{l+m+mi}{0}\PY{p}{)}\PY{p}{;} \PY{k+kn}{from} \PY{n+nn}{gym}\PY{n+nn}{.}\PY{n+nn}{spaces} \PY{k}{import} \PY{n}{prng}\PY{p}{;} \PY{n}{prng}\PY{o}{.}\PY{n}{seed}\PY{p}{(}\PY{l+m+mi}{10}\PY{p}{)}
        \PY{c+c1}{\PYZsh{} Generate the episode}
        \PY{n}{env}\PY{o}{.}\PY{n}{reset}\PY{p}{(}\PY{p}{)}
        \PY{k}{for} \PY{n}{t} \PY{o+ow}{in} \PY{n+nb}{range}\PY{p}{(}\PY{l+m+mi}{100}\PY{p}{)}\PY{p}{:}
            \PY{n}{env}\PY{o}{.}\PY{n}{render}\PY{p}{(}\PY{p}{)}
            \PY{n}{a} \PY{o}{=} \PY{n}{env}\PY{o}{.}\PY{n}{action\PYZus{}space}\PY{o}{.}\PY{n}{sample}\PY{p}{(}\PY{p}{)}
            \PY{n}{ob}\PY{p}{,} \PY{n}{rew}\PY{p}{,} \PY{n}{done}\PY{p}{,} \PY{n}{\PYZus{}} \PY{o}{=} \PY{n}{env}\PY{o}{.}\PY{n}{step}\PY{p}{(}\PY{n}{a}\PY{p}{)}
            \PY{k}{if} \PY{n}{done}\PY{p}{:}
                \PY{k}{break}
        \PY{k}{assert} \PY{n}{done}
        \PY{n}{env}\PY{o}{.}\PY{n}{render}\PY{p}{(}\PY{p}{)}\PY{p}{;}
\end{Verbatim}


    \begin{Verbatim}[commandchars=\\\{\}]

\setlength{\fboxsep}{0pt}\colorbox{ansi-red}{S\strut}FFF
FHFH
FFFH
HFFG
  (Down)
S\setlength{\fboxsep}{0pt}\colorbox{ansi-red}{F\strut}FF
FHFH
FFFH
HFFG
  (Down)
SFFF
F\setlength{\fboxsep}{0pt}\colorbox{ansi-red}{H\strut}FH
FFFH
HFFG

    \end{Verbatim}

    In the episode~above, the agent falls into a hole after two timesteps.
Also note the stochasticity-\/-on the first step, the DOWN action is
selected, but the agent moves to the right.

We extract the relevant information from the gym Env into the MDP class
below. The \texttt{env} object won't be used any further, we'll just use
the \texttt{mdp} object.

    \begin{Verbatim}[commandchars=\\\{\}]
{\color{incolor}In [{\color{incolor}4}]:} \PY{k}{class} \PY{n+nc}{MDP}\PY{p}{(}\PY{n+nb}{object}\PY{p}{)}\PY{p}{:}
            \PY{k}{def} \PY{n+nf}{\PYZus{}\PYZus{}init\PYZus{}\PYZus{}}\PY{p}{(}\PY{n+nb+bp}{self}\PY{p}{,} \PY{n}{P}\PY{p}{,} \PY{n}{nS}\PY{p}{,} \PY{n}{nA}\PY{p}{,} \PY{n}{desc}\PY{o}{=}\PY{k+kc}{None}\PY{p}{)}\PY{p}{:}
                \PY{n+nb+bp}{self}\PY{o}{.}\PY{n}{P} \PY{o}{=} \PY{n}{P} \PY{c+c1}{\PYZsh{} state transition and reward probabilities, explained below}
                \PY{n+nb+bp}{self}\PY{o}{.}\PY{n}{nS} \PY{o}{=} \PY{n}{nS} \PY{c+c1}{\PYZsh{} number of states}
                \PY{n+nb+bp}{self}\PY{o}{.}\PY{n}{nA} \PY{o}{=} \PY{n}{nA} \PY{c+c1}{\PYZsh{} number of actions}
                \PY{n+nb+bp}{self}\PY{o}{.}\PY{n}{desc} \PY{o}{=} \PY{n}{desc} \PY{c+c1}{\PYZsh{} 2D array specifying what each grid cell means (used for plotting)}
        \PY{n}{mdp} \PY{o}{=} \PY{n}{MDP}\PY{p}{(} \PY{p}{\PYZob{}}\PY{n}{s} \PY{p}{:} \PY{p}{\PYZob{}}\PY{n}{a} \PY{p}{:} \PY{p}{[}\PY{n}{tup}\PY{p}{[}\PY{p}{:}\PY{l+m+mi}{3}\PY{p}{]} \PY{k}{for} \PY{n}{tup} \PY{o+ow}{in} \PY{n}{tups}\PY{p}{]} \PY{k}{for} \PY{p}{(}\PY{n}{a}\PY{p}{,} \PY{n}{tups}\PY{p}{)} \PY{o+ow}{in} \PY{n}{a2d}\PY{o}{.}\PY{n}{items}\PY{p}{(}\PY{p}{)}\PY{p}{\PYZcb{}} \PY{k}{for} \PY{p}{(}\PY{n}{s}\PY{p}{,} \PY{n}{a2d}\PY{p}{)} \PY{o+ow}{in} \PY{n}{env}\PY{o}{.}\PY{n}{P}\PY{o}{.}\PY{n}{items}\PY{p}{(}\PY{p}{)}\PY{p}{\PYZcb{}}\PY{p}{,} \PY{n}{env}\PY{o}{.}\PY{n}{nS}\PY{p}{,} \PY{n}{env}\PY{o}{.}\PY{n}{nA}\PY{p}{,} \PY{n}{env}\PY{o}{.}\PY{n}{desc}\PY{p}{)}
        
        
        \PY{n+nb}{print}\PY{p}{(}\PY{l+s+s2}{\PYZdq{}}\PY{l+s+s2}{mdp.P is a two\PYZhy{}level dict where the first key is the state and the second key is the action.}\PY{l+s+s2}{\PYZdq{}}\PY{p}{)}
        \PY{n+nb}{print}\PY{p}{(}\PY{l+s+s2}{\PYZdq{}}\PY{l+s+s2}{The 2D grid cells are associated with indices [0, 1, 2, ..., 15] from left to right and top to down, as in}\PY{l+s+s2}{\PYZdq{}}\PY{p}{)}
        \PY{n+nb}{print}\PY{p}{(}\PY{n}{np}\PY{o}{.}\PY{n}{arange}\PY{p}{(}\PY{l+m+mi}{16}\PY{p}{)}\PY{o}{.}\PY{n}{reshape}\PY{p}{(}\PY{l+m+mi}{4}\PY{p}{,}\PY{l+m+mi}{4}\PY{p}{)}\PY{p}{)}
        \PY{n+nb}{print}\PY{p}{(}\PY{l+s+s2}{\PYZdq{}}\PY{l+s+s2}{Action indices [0, 1, 2, 3] correspond to West, South, East and North.}\PY{l+s+s2}{\PYZdq{}}\PY{p}{)}
        \PY{n+nb}{print}\PY{p}{(}\PY{l+s+s2}{\PYZdq{}}\PY{l+s+s2}{mdp.P[state][action] is a list of tuples (probability, nextstate, reward).}\PY{l+s+se}{\PYZbs{}n}\PY{l+s+s2}{\PYZdq{}}\PY{p}{)}
        \PY{n+nb}{print}\PY{p}{(}\PY{l+s+s2}{\PYZdq{}}\PY{l+s+s2}{For example, state 0 is the initial state, and the transition information for s=0, a=0 is }\PY{l+s+se}{\PYZbs{}n}\PY{l+s+s2}{P[0][0] =}\PY{l+s+s2}{\PYZdq{}}\PY{p}{,} \PY{n}{mdp}\PY{o}{.}\PY{n}{P}\PY{p}{[}\PY{l+m+mi}{0}\PY{p}{]}\PY{p}{[}\PY{l+m+mi}{0}\PY{p}{]}\PY{p}{,} \PY{l+s+s2}{\PYZdq{}}\PY{l+s+se}{\PYZbs{}n}\PY{l+s+s2}{\PYZdq{}}\PY{p}{)}
        \PY{n+nb}{print}\PY{p}{(}\PY{l+s+s2}{\PYZdq{}}\PY{l+s+s2}{As another example, state 5 corresponds to a hole in the ice, in which all actions lead to the same state with probability 1 and reward 0.}\PY{l+s+s2}{\PYZdq{}}\PY{p}{)}
        \PY{k}{for} \PY{n}{i} \PY{o+ow}{in} \PY{n+nb}{range}\PY{p}{(}\PY{l+m+mi}{4}\PY{p}{)}\PY{p}{:}
            \PY{n+nb}{print}\PY{p}{(}\PY{l+s+s2}{\PYZdq{}}\PY{l+s+s2}{P[5][}\PY{l+s+si}{\PYZpc{}i}\PY{l+s+s2}{] =}\PY{l+s+s2}{\PYZdq{}} \PY{o}{\PYZpc{}} \PY{n}{i}\PY{p}{,} \PY{n}{mdp}\PY{o}{.}\PY{n}{P}\PY{p}{[}\PY{l+m+mi}{5}\PY{p}{]}\PY{p}{[}\PY{n}{i}\PY{p}{]}\PY{p}{)}
\end{Verbatim}


    \begin{Verbatim}[commandchars=\\\{\}]
mdp.P is a two-level dict where the first key is the state and the second key is the action.
The 2D grid cells are associated with indices [0, 1, 2, {\ldots}, 15] from left to right and top to down, as in
[[ 0  1  2  3]
 [ 4  5  6  7]
 [ 8  9 10 11]
 [12 13 14 15]]
Action indices [0, 1, 2, 3] correspond to West, South, East and North.
mdp.P[state][action] is a list of tuples (probability, nextstate, reward).

For example, state 0 is the initial state, and the transition information for s=0, a=0 is 
P[0][0] = [(0.1, 0, 0.0), (0.8, 0, 0.0), (0.1, 4, 0.0)] 

As another example, state 5 corresponds to a hole in the ice, in which all actions lead to the same state with probability 1 and reward 0.
P[5][0] = [(1.0, 5, 0)]
P[5][1] = [(1.0, 5, 0)]
P[5][2] = [(1.0, 5, 0)]
P[5][3] = [(1.0, 5, 0)]

    \end{Verbatim}

    \subsection{Part 1: Value Iteration}\label{part-1-value-iteration}

    \subsubsection{Problem 1: implement value
iteration}\label{problem-1-implement-value-iteration}

In this problem, you'll implement value iteration, which has the
following pseudocode:

We additionally define the sequence of greedy policies
\(\pi^{(0)}, \pi^{(1)}, \dots, \pi^{(n-1)}\), where
\[\pi^{(i)}(s) = \arg \max_a \sum_{s'} P(s,a,s') [ R(s,a,s') + \gamma V^{(i)}(s')]\]

Your code will return two lists: \([V^{(0)}, V^{(1)}, \dots, V^{(n)}]\)
and \([\pi^{(0)}, \pi^{(1)}, \dots, \pi^{(n-1)}]\)

To ensure that you get the same policies as the reference solution,
choose the lower-index action to break ties in \(\arg \max_a\). This is
done automatically by np.argmax. This will only affect the "\# chg
actions" printout below-\/-it won't affect the values computed.

Warning: make a copy of your value function each iteration and use that
copy for the update-\/-don't update your value function in place.
Updating in-place is also a valid algorithm, sometimes called
Gauss-Seidel value iteration or asynchronous value iteration, but it
will cause you to get different results than our reference solution
(which in turn will mean that our testing code won't be able to help in
verifying your code).

    \begin{Verbatim}[commandchars=\\\{\}]
{\color{incolor}In [{\color{incolor}11}]:} \PY{k}{def} \PY{n+nf}{value\PYZus{}iteration}\PY{p}{(}\PY{n}{mdp}\PY{p}{,} \PY{n}{gamma}\PY{p}{,} \PY{n}{nIt}\PY{p}{,} \PY{n}{grade\PYZus{}print}\PY{o}{=}\PY{n+nb}{print}\PY{p}{)}\PY{p}{:}
             \PY{l+s+sd}{\PYZdq{}\PYZdq{}\PYZdq{}}
         \PY{l+s+sd}{    Inputs:}
         \PY{l+s+sd}{        mdp: MDP}
         \PY{l+s+sd}{        gamma: discount factor}
         \PY{l+s+sd}{        nIt: number of iterations, corresponding to n above}
         \PY{l+s+sd}{    Outputs:}
         \PY{l+s+sd}{        (value\PYZus{}functions, policies)}
         \PY{l+s+sd}{        }
         \PY{l+s+sd}{    len(value\PYZus{}functions) == nIt+1 and len(policies) == nIt}
         \PY{l+s+sd}{    \PYZdq{}\PYZdq{}\PYZdq{}}
             
             
             \PY{n}{grade\PYZus{}print}\PY{p}{(}\PY{l+s+s2}{\PYZdq{}}\PY{l+s+s2}{Iteration | max|V\PYZhy{}Vprev| | \PYZsh{} chg actions | V[0]}\PY{l+s+s2}{\PYZdq{}}\PY{p}{)}
             \PY{n}{grade\PYZus{}print}\PY{p}{(}\PY{l+s+s2}{\PYZdq{}}\PY{l+s+s2}{\PYZhy{}\PYZhy{}\PYZhy{}\PYZhy{}\PYZhy{}\PYZhy{}\PYZhy{}\PYZhy{}\PYZhy{}\PYZhy{}+\PYZhy{}\PYZhy{}\PYZhy{}\PYZhy{}\PYZhy{}\PYZhy{}\PYZhy{}\PYZhy{}\PYZhy{}\PYZhy{}\PYZhy{}\PYZhy{}\PYZhy{}\PYZhy{}+\PYZhy{}\PYZhy{}\PYZhy{}\PYZhy{}\PYZhy{}\PYZhy{}\PYZhy{}\PYZhy{}\PYZhy{}\PYZhy{}\PYZhy{}\PYZhy{}\PYZhy{}\PYZhy{}\PYZhy{}+\PYZhy{}\PYZhy{}\PYZhy{}\PYZhy{}\PYZhy{}\PYZhy{}\PYZhy{}\PYZhy{}\PYZhy{}}\PY{l+s+s2}{\PYZdq{}}\PY{p}{)}
             \PY{n}{Vs} \PY{o}{=} \PY{p}{[}\PY{n}{np}\PY{o}{.}\PY{n}{zeros}\PY{p}{(}\PY{n}{mdp}\PY{o}{.}\PY{n}{nS}\PY{p}{)}\PY{p}{]} \PY{c+c1}{\PYZsh{} list of value functions contains the initial value function V\PYZca{}\PYZob{}(0)\PYZcb{}, which is zero}
             \PY{n}{pis} \PY{o}{=} \PY{p}{[}\PY{p}{]}
              
             \PY{k}{for} \PY{n}{it} \PY{o+ow}{in} \PY{n+nb}{range}\PY{p}{(}\PY{n}{nIt}\PY{p}{)}\PY{p}{:}
                 \PY{n}{oldpi} \PY{o}{=} \PY{n}{pis}\PY{p}{[}\PY{o}{\PYZhy{}}\PY{l+m+mi}{1}\PY{p}{]} \PY{k}{if} \PY{n+nb}{len}\PY{p}{(}\PY{n}{pis}\PY{p}{)} \PY{o}{\PYZgt{}} \PY{l+m+mi}{0} \PY{k}{else} \PY{k+kc}{None} \PY{c+c1}{\PYZsh{} \PYZbs{}pi\PYZca{}\PYZob{}(it)\PYZcb{} = Greedy[V\PYZca{}\PYZob{}(it\PYZhy{}1)\PYZcb{}]. Just used for printout}
                 \PY{n}{Vprev} \PY{o}{=} \PY{n}{Vs}\PY{p}{[}\PY{o}{\PYZhy{}}\PY{l+m+mi}{1}\PY{p}{]} \PY{c+c1}{\PYZsh{} V\PYZca{}\PYZob{}(it)\PYZcb{}}
                 
                 \PY{c+c1}{\PYZsh{} Your code should fill in meaningful values for the following two variables}
                 \PY{c+c1}{\PYZsh{} pi: greedy policy for Vprev (not V), }
                 \PY{c+c1}{\PYZsh{}     corresponding to the math above: \PYZbs{}pi\PYZca{}\PYZob{}(it)\PYZcb{} = Greedy[V\PYZca{}\PYZob{}(it)\PYZcb{}]}
                 \PY{c+c1}{\PYZsh{}     ** it needs to be numpy array of ints **}
                 \PY{c+c1}{\PYZsh{} V: bellman backup on Vprev}
                 \PY{c+c1}{\PYZsh{}     corresponding to the math above: V\PYZca{}\PYZob{}(it+1)\PYZcb{} = T[V\PYZca{}\PYZob{}(it)\PYZcb{}]}
                 \PY{c+c1}{\PYZsh{}     ** numpy array of floats **}
                 
                 \PY{n}{V} \PY{o}{=} \PY{n}{np}\PY{o}{.}\PY{n}{max}\PY{p}{(}\PY{p}{)} \PY{c+c1}{\PYZsh{} REPLACE THIS LINE WITH YOUR CODE}
                 \PY{n}{pi} \PY{o}{=} \PY{n}{oldpi} \PY{c+c1}{\PYZsh{} REPLACE THIS LINE WITH YOUR CODE}
                 
                 \PY{n}{max\PYZus{}diff} \PY{o}{=} \PY{n}{np}\PY{o}{.}\PY{n}{abs}\PY{p}{(}\PY{n}{V} \PY{o}{\PYZhy{}} \PY{n}{Vprev}\PY{p}{)}\PY{o}{.}\PY{n}{max}\PY{p}{(}\PY{p}{)}
                 \PY{n}{nChgActions}\PY{o}{=}\PY{l+s+s2}{\PYZdq{}}\PY{l+s+s2}{N/A}\PY{l+s+s2}{\PYZdq{}} \PY{k}{if} \PY{n}{oldpi} \PY{o+ow}{is} \PY{k+kc}{None} \PY{k}{else} \PY{p}{(}\PY{n}{pi} \PY{o}{!=} \PY{n}{oldpi}\PY{p}{)}\PY{o}{.}\PY{n}{sum}\PY{p}{(}\PY{p}{)}
                 \PY{n}{grade\PYZus{}print}\PY{p}{(}\PY{l+s+s2}{\PYZdq{}}\PY{l+s+si}{\PYZpc{}4i}\PY{l+s+s2}{      | }\PY{l+s+si}{\PYZpc{}6.5f}\PY{l+s+s2}{      | }\PY{l+s+si}{\PYZpc{}4s}\PY{l+s+s2}{          | }\PY{l+s+si}{\PYZpc{}5.3f}\PY{l+s+s2}{\PYZdq{}}\PY{o}{\PYZpc{}}\PY{p}{(}\PY{n}{it}\PY{p}{,} \PY{n}{max\PYZus{}diff}\PY{p}{,} \PY{n}{nChgActions}\PY{p}{,} \PY{n}{V}\PY{p}{[}\PY{l+m+mi}{0}\PY{p}{]}\PY{p}{)}\PY{p}{)}
                 \PY{n}{Vs}\PY{o}{.}\PY{n}{append}\PY{p}{(}\PY{n}{V}\PY{p}{)}
                 \PY{n}{pis}\PY{o}{.}\PY{n}{append}\PY{p}{(}\PY{n}{pi}\PY{p}{)}
             \PY{k}{return} \PY{n}{Vs}\PY{p}{,} \PY{n}{pis}
         
         \PY{n}{GAMMA} \PY{o}{=} \PY{l+m+mf}{0.95} \PY{c+c1}{\PYZsh{} we\PYZsq{}ll be using this same value in subsequent problems}
         
         
         \PY{c+c1}{\PYZsh{} The following is the output of a correct implementation; when}
         \PY{c+c1}{\PYZsh{}   this code block is run, your implementation\PYZsq{}s print output will be}
         \PY{c+c1}{\PYZsh{}   compared with expected output.}
         \PY{c+c1}{\PYZsh{}   (incorrect line in red background with correct line printed side by side to help you debug)}
         \PY{n}{expected\PYZus{}output} \PY{o}{=} \PY{l+s+s2}{\PYZdq{}\PYZdq{}\PYZdq{}}\PY{l+s+s2}{Iteration | max|V\PYZhy{}Vprev| | \PYZsh{} chg actions | V[0]}
         \PY{l+s+s2}{\PYZhy{}\PYZhy{}\PYZhy{}\PYZhy{}\PYZhy{}\PYZhy{}\PYZhy{}\PYZhy{}\PYZhy{}\PYZhy{}+\PYZhy{}\PYZhy{}\PYZhy{}\PYZhy{}\PYZhy{}\PYZhy{}\PYZhy{}\PYZhy{}\PYZhy{}\PYZhy{}\PYZhy{}\PYZhy{}\PYZhy{}\PYZhy{}+\PYZhy{}\PYZhy{}\PYZhy{}\PYZhy{}\PYZhy{}\PYZhy{}\PYZhy{}\PYZhy{}\PYZhy{}\PYZhy{}\PYZhy{}\PYZhy{}\PYZhy{}\PYZhy{}\PYZhy{}+\PYZhy{}\PYZhy{}\PYZhy{}\PYZhy{}\PYZhy{}\PYZhy{}\PYZhy{}\PYZhy{}\PYZhy{}}
         \PY{l+s+s2}{   0      | 0.80000      |  N/A          | 0.000}
         \PY{l+s+s2}{   1      | 0.60800      |    2          | 0.000}
         \PY{l+s+s2}{   2      | 0.51984      |    2          | 0.000}
         \PY{l+s+s2}{   3      | 0.39508      |    2          | 0.000}
         \PY{l+s+s2}{   4      | 0.30026      |    2          | 0.000}
         \PY{l+s+s2}{   5      | 0.25355      |    1          | 0.254}
         \PY{l+s+s2}{   6      | 0.10478      |    0          | 0.345}
         \PY{l+s+s2}{   7      | 0.09657      |    0          | 0.442}
         \PY{l+s+s2}{   8      | 0.03656      |    0          | 0.478}
         \PY{l+s+s2}{   9      | 0.02772      |    0          | 0.506}
         \PY{l+s+s2}{  10      | 0.01111      |    0          | 0.517}
         \PY{l+s+s2}{  11      | 0.00735      |    0          | 0.524}
         \PY{l+s+s2}{  12      | 0.00310      |    0          | 0.527}
         \PY{l+s+s2}{  13      | 0.00190      |    0          | 0.529}
         \PY{l+s+s2}{  14      | 0.00083      |    0          | 0.530}
         \PY{l+s+s2}{  15      | 0.00049      |    0          | 0.531}
         \PY{l+s+s2}{  16      | 0.00022      |    0          | 0.531}
         \PY{l+s+s2}{  17      | 0.00013      |    0          | 0.531}
         \PY{l+s+s2}{  18      | 0.00006      |    0          | 0.531}
         \PY{l+s+s2}{  19      | 0.00003      |    0          | 0.531}\PY{l+s+s2}{\PYZdq{}\PYZdq{}\PYZdq{}}
         \PY{n}{Vs\PYZus{}VI}\PY{p}{,} \PY{n}{pis\PYZus{}VI} \PY{o}{=} \PY{n}{value\PYZus{}iteration}\PY{p}{(}\PY{n}{mdp}\PY{p}{,} \PY{n}{gamma}\PY{o}{=}\PY{n}{GAMMA}\PY{p}{,} \PY{n}{nIt}\PY{o}{=}\PY{l+m+mi}{20}\PY{p}{,} \PY{n}{grade\PYZus{}print}\PY{o}{=}\PY{n}{make\PYZus{}grader}\PY{p}{(}\PY{n}{expected\PYZus{}output}\PY{p}{)}\PY{p}{)}
\end{Verbatim}


    \begin{Verbatim}[commandchars=\\\{\}]
16
Iteration | max|V-Vprev| | \# chg actions | V[0]
----------+--------------+---------------+---------
\setlength{\fboxsep}{0pt}\colorbox{ansi-red}{   0      | 0.00000      |  N/A          | 0.000\strut} *** Expected: \setlength{\fboxsep}{0pt}\colorbox{ansi-green}{   0      | 0.80000      |  N/A          | 0.000\strut}
\setlength{\fboxsep}{0pt}\colorbox{ansi-red}{   1      | 0.00000      |  N/A          | 0.000\strut} *** Expected: \setlength{\fboxsep}{0pt}\colorbox{ansi-green}{   1      | 0.60800      |    2          | 0.000\strut}
\setlength{\fboxsep}{0pt}\colorbox{ansi-red}{   2      | 0.00000      |  N/A          | 0.000\strut} *** Expected: \setlength{\fboxsep}{0pt}\colorbox{ansi-green}{   2      | 0.51984      |    2          | 0.000\strut}
\setlength{\fboxsep}{0pt}\colorbox{ansi-red}{   3      | 0.00000      |  N/A          | 0.000\strut} *** Expected: \setlength{\fboxsep}{0pt}\colorbox{ansi-green}{   3      | 0.39508      |    2          | 0.000\strut}
\setlength{\fboxsep}{0pt}\colorbox{ansi-red}{   4      | 0.00000      |  N/A          | 0.000\strut} *** Expected: \setlength{\fboxsep}{0pt}\colorbox{ansi-green}{   4      | 0.30026      |    2          | 0.000\strut}
\setlength{\fboxsep}{0pt}\colorbox{ansi-red}{   5      | 0.00000      |  N/A          | 0.000\strut} *** Expected: \setlength{\fboxsep}{0pt}\colorbox{ansi-green}{   5      | 0.25355      |    1          | 0.254\strut}
\setlength{\fboxsep}{0pt}\colorbox{ansi-red}{   6      | 0.00000      |  N/A          | 0.000\strut} *** Expected: \setlength{\fboxsep}{0pt}\colorbox{ansi-green}{   6      | 0.10478      |    0          | 0.345\strut}
\setlength{\fboxsep}{0pt}\colorbox{ansi-red}{   7      | 0.00000      |  N/A          | 0.000\strut} *** Expected: \setlength{\fboxsep}{0pt}\colorbox{ansi-green}{   7      | 0.09657      |    0          | 0.442\strut}
\setlength{\fboxsep}{0pt}\colorbox{ansi-red}{   8      | 0.00000      |  N/A          | 0.000\strut} *** Expected: \setlength{\fboxsep}{0pt}\colorbox{ansi-green}{   8      | 0.03656      |    0          | 0.478\strut}
\setlength{\fboxsep}{0pt}\colorbox{ansi-red}{   9      | 0.00000      |  N/A          | 0.000\strut} *** Expected: \setlength{\fboxsep}{0pt}\colorbox{ansi-green}{   9      | 0.02772      |    0          | 0.506\strut}
\setlength{\fboxsep}{0pt}\colorbox{ansi-red}{  10      | 0.00000      |  N/A          | 0.000\strut} *** Expected: \setlength{\fboxsep}{0pt}\colorbox{ansi-green}{  10      | 0.01111      |    0          | 0.517\strut}
\setlength{\fboxsep}{0pt}\colorbox{ansi-red}{  11      | 0.00000      |  N/A          | 0.000\strut} *** Expected: \setlength{\fboxsep}{0pt}\colorbox{ansi-green}{  11      | 0.00735      |    0          | 0.524\strut}
\setlength{\fboxsep}{0pt}\colorbox{ansi-red}{  12      | 0.00000      |  N/A          | 0.000\strut} *** Expected: \setlength{\fboxsep}{0pt}\colorbox{ansi-green}{  12      | 0.00310      |    0          | 0.527\strut}
\setlength{\fboxsep}{0pt}\colorbox{ansi-red}{  13      | 0.00000      |  N/A          | 0.000\strut} *** Expected: \setlength{\fboxsep}{0pt}\colorbox{ansi-green}{  13      | 0.00190      |    0          | 0.529\strut}
\setlength{\fboxsep}{0pt}\colorbox{ansi-red}{  14      | 0.00000      |  N/A          | 0.000\strut} *** Expected: \setlength{\fboxsep}{0pt}\colorbox{ansi-green}{  14      | 0.00083      |    0          | 0.530\strut}
\setlength{\fboxsep}{0pt}\colorbox{ansi-red}{  15      | 0.00000      |  N/A          | 0.000\strut} *** Expected: \setlength{\fboxsep}{0pt}\colorbox{ansi-green}{  15      | 0.00049      |    0          | 0.531\strut}
\setlength{\fboxsep}{0pt}\colorbox{ansi-red}{  16      | 0.00000      |  N/A          | 0.000\strut} *** Expected: \setlength{\fboxsep}{0pt}\colorbox{ansi-green}{  16      | 0.00022      |    0          | 0.531\strut}
\setlength{\fboxsep}{0pt}\colorbox{ansi-red}{  17      | 0.00000      |  N/A          | 0.000\strut} *** Expected: \setlength{\fboxsep}{0pt}\colorbox{ansi-green}{  17      | 0.00013      |    0          | 0.531\strut}
\setlength{\fboxsep}{0pt}\colorbox{ansi-red}{  18      | 0.00000      |  N/A          | 0.000\strut} *** Expected: \setlength{\fboxsep}{0pt}\colorbox{ansi-green}{  18      | 0.00006      |    0          | 0.531\strut}
\setlength{\fboxsep}{0pt}\colorbox{ansi-red}{  19      | 0.00000      |  N/A          | 0.000\strut} *** Expected: \setlength{\fboxsep}{0pt}\colorbox{ansi-green}{  19      | 0.00003      |    0          | 0.531\strut}
Test failed

    \end{Verbatim}

    Below, we've illustrated the progress of value iteration. Your optimal
actions are shown by arrows. At the bottom, the value of the different
states are plotted.

    \begin{Verbatim}[commandchars=\\\{\}]
{\color{incolor}In [{\color{incolor}6}]:} \PY{k}{for} \PY{p}{(}\PY{n}{V}\PY{p}{,} \PY{n}{pi}\PY{p}{)} \PY{o+ow}{in} \PY{n+nb}{zip}\PY{p}{(}\PY{n}{Vs\PYZus{}VI}\PY{p}{[}\PY{p}{:}\PY{l+m+mi}{10}\PY{p}{]}\PY{p}{,} \PY{n}{pis\PYZus{}VI}\PY{p}{[}\PY{p}{:}\PY{l+m+mi}{10}\PY{p}{]}\PY{p}{)}\PY{p}{:}
            \PY{n}{plt}\PY{o}{.}\PY{n}{figure}\PY{p}{(}\PY{n}{figsize}\PY{o}{=}\PY{p}{(}\PY{l+m+mi}{3}\PY{p}{,}\PY{l+m+mi}{3}\PY{p}{)}\PY{p}{)}
            \PY{n}{plt}\PY{o}{.}\PY{n}{imshow}\PY{p}{(}\PY{n}{V}\PY{o}{.}\PY{n}{reshape}\PY{p}{(}\PY{l+m+mi}{4}\PY{p}{,}\PY{l+m+mi}{4}\PY{p}{)}\PY{p}{,} \PY{n}{cmap}\PY{o}{=}\PY{l+s+s1}{\PYZsq{}}\PY{l+s+s1}{gray}\PY{l+s+s1}{\PYZsq{}}\PY{p}{,} \PY{n}{interpolation}\PY{o}{=}\PY{l+s+s1}{\PYZsq{}}\PY{l+s+s1}{none}\PY{l+s+s1}{\PYZsq{}}\PY{p}{,} \PY{n}{clim}\PY{o}{=}\PY{p}{(}\PY{l+m+mi}{0}\PY{p}{,}\PY{l+m+mi}{1}\PY{p}{)}\PY{p}{)}
            \PY{n}{ax} \PY{o}{=} \PY{n}{plt}\PY{o}{.}\PY{n}{gca}\PY{p}{(}\PY{p}{)}
            \PY{n}{ax}\PY{o}{.}\PY{n}{set\PYZus{}xticks}\PY{p}{(}\PY{n}{np}\PY{o}{.}\PY{n}{arange}\PY{p}{(}\PY{l+m+mi}{4}\PY{p}{)}\PY{o}{\PYZhy{}}\PY{o}{.}\PY{l+m+mi}{5}\PY{p}{)}
            \PY{n}{ax}\PY{o}{.}\PY{n}{set\PYZus{}yticks}\PY{p}{(}\PY{n}{np}\PY{o}{.}\PY{n}{arange}\PY{p}{(}\PY{l+m+mi}{4}\PY{p}{)}\PY{o}{\PYZhy{}}\PY{o}{.}\PY{l+m+mi}{5}\PY{p}{)}
            \PY{n}{ax}\PY{o}{.}\PY{n}{set\PYZus{}xticklabels}\PY{p}{(}\PY{p}{[}\PY{p}{]}\PY{p}{)}
            \PY{n}{ax}\PY{o}{.}\PY{n}{set\PYZus{}yticklabels}\PY{p}{(}\PY{p}{[}\PY{p}{]}\PY{p}{)}
            \PY{n}{Y}\PY{p}{,} \PY{n}{X} \PY{o}{=} \PY{n}{np}\PY{o}{.}\PY{n}{mgrid}\PY{p}{[}\PY{l+m+mi}{0}\PY{p}{:}\PY{l+m+mi}{4}\PY{p}{,} \PY{l+m+mi}{0}\PY{p}{:}\PY{l+m+mi}{4}\PY{p}{]}
            \PY{n}{a2uv} \PY{o}{=} \PY{p}{\PYZob{}}\PY{l+m+mi}{0}\PY{p}{:} \PY{p}{(}\PY{o}{\PYZhy{}}\PY{l+m+mi}{1}\PY{p}{,} \PY{l+m+mi}{0}\PY{p}{)}\PY{p}{,} \PY{l+m+mi}{1}\PY{p}{:}\PY{p}{(}\PY{l+m+mi}{0}\PY{p}{,} \PY{o}{\PYZhy{}}\PY{l+m+mi}{1}\PY{p}{)}\PY{p}{,} \PY{l+m+mi}{2}\PY{p}{:}\PY{p}{(}\PY{l+m+mi}{1}\PY{p}{,}\PY{l+m+mi}{0}\PY{p}{)}\PY{p}{,} \PY{l+m+mi}{3}\PY{p}{:}\PY{p}{(}\PY{o}{\PYZhy{}}\PY{l+m+mi}{1}\PY{p}{,} \PY{l+m+mi}{0}\PY{p}{)}\PY{p}{\PYZcb{}}
            \PY{n}{Pi} \PY{o}{=} \PY{n}{pi}\PY{o}{.}\PY{n}{reshape}\PY{p}{(}\PY{l+m+mi}{4}\PY{p}{,}\PY{l+m+mi}{4}\PY{p}{)}
            \PY{k}{for} \PY{n}{y} \PY{o+ow}{in} \PY{n+nb}{range}\PY{p}{(}\PY{l+m+mi}{4}\PY{p}{)}\PY{p}{:}
                \PY{k}{for} \PY{n}{x} \PY{o+ow}{in} \PY{n+nb}{range}\PY{p}{(}\PY{l+m+mi}{4}\PY{p}{)}\PY{p}{:}
                    \PY{n}{a} \PY{o}{=} \PY{n}{Pi}\PY{p}{[}\PY{n}{y}\PY{p}{,} \PY{n}{x}\PY{p}{]}
                    \PY{n}{u}\PY{p}{,} \PY{n}{v} \PY{o}{=} \PY{n}{a2uv}\PY{p}{[}\PY{n}{a}\PY{p}{]}
                    \PY{n}{plt}\PY{o}{.}\PY{n}{arrow}\PY{p}{(}\PY{n}{x}\PY{p}{,} \PY{n}{y}\PY{p}{,}\PY{n}{u}\PY{o}{*}\PY{o}{.}\PY{l+m+mi}{3}\PY{p}{,} \PY{o}{\PYZhy{}}\PY{n}{v}\PY{o}{*}\PY{o}{.}\PY{l+m+mi}{3}\PY{p}{,} \PY{n}{color}\PY{o}{=}\PY{l+s+s1}{\PYZsq{}}\PY{l+s+s1}{m}\PY{l+s+s1}{\PYZsq{}}\PY{p}{,} \PY{n}{head\PYZus{}width}\PY{o}{=}\PY{l+m+mf}{0.1}\PY{p}{,} \PY{n}{head\PYZus{}length}\PY{o}{=}\PY{l+m+mf}{0.1}\PY{p}{)} 
                    \PY{n}{plt}\PY{o}{.}\PY{n}{text}\PY{p}{(}\PY{n}{x}\PY{p}{,} \PY{n}{y}\PY{p}{,} \PY{n+nb}{str}\PY{p}{(}\PY{n}{env}\PY{o}{.}\PY{n}{desc}\PY{p}{[}\PY{n}{y}\PY{p}{,}\PY{n}{x}\PY{p}{]}\PY{o}{.}\PY{n}{item}\PY{p}{(}\PY{p}{)}\PY{o}{.}\PY{n}{decode}\PY{p}{(}\PY{p}{)}\PY{p}{)}\PY{p}{,}
                             \PY{n}{color}\PY{o}{=}\PY{l+s+s1}{\PYZsq{}}\PY{l+s+s1}{g}\PY{l+s+s1}{\PYZsq{}}\PY{p}{,} \PY{n}{size}\PY{o}{=}\PY{l+m+mi}{12}\PY{p}{,}  \PY{n}{verticalalignment}\PY{o}{=}\PY{l+s+s1}{\PYZsq{}}\PY{l+s+s1}{center}\PY{l+s+s1}{\PYZsq{}}\PY{p}{,}
                             \PY{n}{horizontalalignment}\PY{o}{=}\PY{l+s+s1}{\PYZsq{}}\PY{l+s+s1}{center}\PY{l+s+s1}{\PYZsq{}}\PY{p}{,} \PY{n}{fontweight}\PY{o}{=}\PY{l+s+s1}{\PYZsq{}}\PY{l+s+s1}{bold}\PY{l+s+s1}{\PYZsq{}}\PY{p}{)}
            \PY{n}{plt}\PY{o}{.}\PY{n}{grid}\PY{p}{(}\PY{n}{color}\PY{o}{=}\PY{l+s+s1}{\PYZsq{}}\PY{l+s+s1}{b}\PY{l+s+s1}{\PYZsq{}}\PY{p}{,} \PY{n}{lw}\PY{o}{=}\PY{l+m+mi}{2}\PY{p}{,} \PY{n}{ls}\PY{o}{=}\PY{l+s+s1}{\PYZsq{}}\PY{l+s+s1}{\PYZhy{}}\PY{l+s+s1}{\PYZsq{}}\PY{p}{)}
        \PY{n}{plt}\PY{o}{.}\PY{n}{figure}\PY{p}{(}\PY{p}{)}
        \PY{n}{plt}\PY{o}{.}\PY{n}{plot}\PY{p}{(}\PY{n}{Vs\PYZus{}VI}\PY{p}{)}
        \PY{n}{plt}\PY{o}{.}\PY{n}{title}\PY{p}{(}\PY{l+s+s2}{\PYZdq{}}\PY{l+s+s2}{Values of different states}\PY{l+s+s2}{\PYZdq{}}\PY{p}{)}\PY{p}{;}
\end{Verbatim}


    \begin{Verbatim}[commandchars=\\\{\}]

        ---------------------------------------------------------------------------

        AttributeError                            Traceback (most recent call last)

        <ipython-input-6-95196b916f92> in <module>()
          9     Y, X = np.mgrid[0:4, 0:4]
         10     a2uv = \{0: (-1, 0), 1:(0, -1), 2:(1,0), 3:(-1, 0)\}
    ---> 11     Pi = pi.reshape(4,4)
         12     for y in range(4):
         13         for x in range(4):


        AttributeError: 'NoneType' object has no attribute 'reshape'

    \end{Verbatim}

    \begin{center}
    \adjustimage{max size={0.9\linewidth}{0.9\paperheight}}{output_11_1.png}
    \end{center}
    { \hspace*{\fill} \\}
    

    % Add a bibliography block to the postdoc
    
    
    
    \end{document}
